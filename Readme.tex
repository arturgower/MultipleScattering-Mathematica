
\documentclass[ 12pt, a4paper]{article}
% Use the option doublespacing or reviewcopy to obtain double line spacing
% \documentclass[doublespacing]{elsart}

% the natbib package allows both number and author-year (Harvard)
% style referencing;
\usepackage{natbib}
% if you use PostScript figures in your article
% use the graphics package for simple commands
%\usepackage{graphics}
% or use the graphicx package for more complicated commands
\usepackage{graphicx}
% or use the epsfig package if you prefer to use the old commands
%\usepackage{epsfig}
% The amssymb package provides various useful mathematical symbols
\usepackage{amssymb}
\usepackage{amsmath}
%\usepackage{amsthm}
\usepackage{mathdots}
%\usepackage{mathaccents}
\evensidemargin=0cm \oddsidemargin=0cm \setlength{\textwidth}{16cm}
\usepackage{setspace}
\usepackage{stmaryrd}
\usepackage{theorem}
\usepackage{hyperref}

\newtheorem{theorem}{Theorem}
\newtheorem{corollary}{Corollary}
\newtheorem{lema}{Lema}
\theorembodyfont{\normalfont}
\newtheorem{definition}{Definition}
\newtheorem{example}{Example}
\newtheorem{solution}{Solution}
\def \proof{\noindent {\bf \emph{Proof:}} }

\newcommand{\be}{\begin{equation}}
\newcommand{\en}{\end{equation}}
\def\bga#1\ega{\begin{gather}#1\end{gather}} % suggested in technote.tex
\def\bgas#1\egas{\begin{gather*}#1\end{gather*}}

\def\bal#1\eal{\begin{align}#1\end{align}} % suggested in technote.tex
\def\bals#1\eals{\begin{align*}#1\end{align*}}

\newcommand{\initial}[1]{{#1}_\circ}
\renewcommand{\thefootnote}{\fnsymbol{footnote}}

\DeclareMathOperator{\sign}{sgn}
\DeclareMathOperator{\divergence}{div}
\DeclareMathOperator{\tr}{tr}
\DeclareMathOperator{\Ord}{\mathcal{O}}
\DeclareMathOperator{\GRAD}{grad}
\DeclareMathOperator{\DIV}{DIV}

\newcommand{\filler}{\hspace*{\fill}}
\newcommand{\ii}{\textrm{i}}
\newcommand{\ee}{\textrm{e}}
\renewcommand{\vec}[1]{\boldsymbol{#1}}
\def \bm#1{\mbox{\boldmath{$#1$}}}   % this is used to write boldface Greek

\graphicspath{{../Images/}}
\graphicspath{{Media/}}

\doublespacing

\setlength{\topmargin}{0cm} \addtolength{\textheight}{2cm}

\begin{document}

\title{\sc{ Brief on the package} \texttt{MultipleScattering2D}}
\author{
Artur L Gower\footnotemark[2]  \\[12pt]
%$^a$ School of Mathematics, Statistics and Applied Mathematics,\\
%National University of Ireland Galway;\\
}
\date{\today}
\maketitle

\footnotetext[2]{School of Mathematics, University of Manchester, Oxford Road, Manchester, M13 9PL, UK.}

\begin{abstract}
Written in Mathematica 10.2, this package calculates the scattered scalar wave, in 2D, from any configurations of cylinders. To see how to to do multiple scattering, and make pretty pictures, go to the folder \url{examples}. Inside the folder this is \url{examples/TwoBodyScattering.nb} for scattering from two cylinders, \url{examples/OneCylinder.nb} for a simpler example of scattering from one cylinder and \url{examples/Source.nb} for some details on sources. At the moment the cylinders all have to be the same and we consider no wave transmission through the cylinders. 
\end{abstract}

\noindent
{\textit{Keywords:}fish }


%%%%%%%%%%%%

\section{Introduction}

Below is a brief about the functions \texttt{AcousticScattering}, \texttt{AcousticImpulse} and \texttt{PlotWaves}. The next sections contains some of theory behind this package.

{\ttfamily
scatteredWave = AcousticScattering[$r_S, \vec r_I, \{t_0, t_{max}, \delta t\}, \{R_{max}, \delta R, \delta \theta\}$,options];
}
returns the scattered wave from one cylinder at $\{0,0\}$, where
\begin{itemize}
\item each element of \texttt{Flatten[scatteredWave, 2]} is of the form \texttt{ \{$t, \theta, r,
    wave[t,\theta,r]$\}}.

\item $r_S$ is the radius of the scatterer, $\vec r_I = \{x_I, y_I \}$ is the centre of the impulse that generates the incident wave. The scatterer is always placed at $\{0,0\}$.
\item The list \{$t_0$, $t_{max}$, $\delta t$\} specifies the range for the time considered, so $\{t_0, t_{max}, \delta t\}$ = \{1, 2, 0.5\} means that the scattered wave will be calculated for $t =1, 1.5$ and $2$. The list $\{t_0, t_{max}, \delta t\}$ can also be replaced by just one time $\{t_0, t_{max}, \delta t\} =t_0$ and then the solution will be given just for $t=t_0$, which is useful if you want to calculate something that might crash the memory.
\item The list $\{R_{max}, \delta R, \delta \theta\}$ specifies the discretization of $r$ and $\theta$ (in cylindrical coordinates), so for $\{R_{max}, \delta R, \delta \theta\} ={2, 0.5, \pi/4}$ and $r_S= 0.5$ means that the solution will be calculated for $r =0.5, 1., 1.5,2$ and $\theta =0, \pi/4, 2\pi/4, \pi$.
\item The \texttt{options} argument is optional but it can be of the form \texttt{options = Sequence["Impulse" -> b, "ImpulsePeriod" -> 2.,
  "PrintChecks" -> True, "Boundary" -> "Neumann",   "FrequencyModes" -> 20, "NAngularModes" -> 7]; }, which would specify the  body forcea $B(x,t) = \delta(\vec x) b(t)$ where $b(t) =0$ for $t<0$ or $t>2.$. The rest is better explained below.
\item The function \texttt{
incidentWave= AcousticImpulse[$t, \{r_{max},\delta r\},$ options]
} generates a list \texttt{ incidentWave =\{$\{0, \varphi^I(0,t)\},\{\delta r, \varphi^I(\delta r,t)\} ,\ldots, \{r_{max}, \varphi^I(r_{max},t)\}$\}; }.
\item To plot the results just run \texttt{PlotWaves[scatteredWave]} or  \texttt{PlotWaves[incidentWave]}.
\end{itemize}


\subsection{Incident wave}

We look to solve the 3D wave equation
\be
\mathcal L \{\varphi \}(\vec x,t) = \frac{1}{c^2} \frac{\partial^2 \varphi}{\partial t^2}(\vec x,t) - \nabla^2 \varphi(\vec x,t)  = \frac{1}{c^2}B(\vec x,t),
\label{eqn:ScalarWaveEquation}
\en
with the conditions
\be
\varphi(\vec x,0^{-}) =0 , \quad \dot \varphi( \vec x,0^{-}) =0 \quad \text{and} \quad  \lim_{\|\vec x\| \to 0} \varphi(\vec x,t) =0,
\en
where $B$ is the body force\footnote{This $B$ is technically only a body force for an elastic SH-wave. The interpretation of $B$ depends on the physical interpretation of $\varphi$.}. To solve this we use the Delta Dirac $\delta(\vec x) = \delta(x)\delta(y)\delta(z) = \delta (r)/(4 \pi r^2)$ if $r$ is the radius of a spherical coordinate system, and first solve the wave equation in spherical coordinates
\be
\frac{1}{c^2}\frac{\partial^2 g}{\partial t^2} -  \frac{1}{r^2} \frac{\partial}{\partial r} \left (r^2 \frac{\partial g}{\partial r} \right )  = \frac{\delta(r)}{4 \pi r^2} \delta (t)
\en
with $g(r,t) =0$ for $t<0$.  This is not trivially solved, as when differentiating $r^{-1}$ a distribution appears on the origin. The solution can be found in p. 92 Achenbach (1973)\footnote{There he changes to spherical coordinates, substitutes $\varphi(r,t) = \Phi(r,t)/r$, and with witchcraft solves the resulting scalar wave equation, picking only the outgoing wave} .
\be
g(r, t) = \frac{1}{4 \pi r} \delta(t - r/c).
\label{eqn:ScalarGreen3D}
\en
%Note replacing $\delta$ for some function $f$ above would give the solution for when $B(x,t) = c^2 f(t) \delta( \vec x)$.
 If we let $B(\vec x,t) = \delta (\vec x) b(t)$, then the solution to Eq.~\eqref{eqn:ScalarWaveEquation} without a scatterer, i.e. for the incident wave, becomes
\begin{multline}
\varphi^I(\vec x,t) = \int g(\vec x - \vec \xi,t -\tau)\frac{1}{c^2} B(\vec \xi, \tau) d \vec \xi d\tau =
\\
 \int g(\vec x - \vec \xi,t -\tau)\frac{1}{c^2} \delta(\vec \xi) b(\tau) d \vec \xi d\tau =  \int \delta(t- \tau - r/c) \frac{b(\tau)}{4 c^2 \pi r}  d\tau = \frac{b(t- r/c)}{4 c^2 \pi r}.
\label{eqn:ScalarPointImpact}
%  \int \frac{1}{4 \pi}  \frac{1}{c^2} B\left ( \sqrt{r^2 + r\xi^2   \mp 2  r r_\xi \cos\theta_\xi \sin\phi_\xi} , t -r_\xi /c \right) r_\xi \sin \phi_\xi d \phi_\xi d \theta_\xi d r_\xi ,
\end{multline}
which is the solution to Eq.~\eqref{eqn:ScalarWaveEquation} because
\[
\mathcal L \{\varphi^I \}(\vec x,t) = \int \mathcal L \{ g \}(\vec x -\vec \xi,t -\tau)\frac{1}{c^2}B(\vec \xi, \tau) d \xi d \tau =  \int  \delta(\vec x -\vec \xi,t -\tau) \frac{1}{c^2} B(\vec \xi, \tau) d \xi d \tau = \frac{1}{c^2}B(\vec x,t).
\]
Let us adopt the Fourier transform convention:
\[
f(t) = \frac{1}{2 \pi} \int_{-\infty}^\infty \hat f(\omega) \ee^{- \ii \omega t} d \omega \;\; \text{and} \;\; \hat f(\omega) = \int_{-\infty}^\infty  f(t) \ee^{\ii \omega t} d t.
\]

If we took one frequency $g(r,t) =  \hat g(r,\omega) \ee^{-\ii \omega t}$ and solved for $\hat g$ with $B = c^2 \ee^{- \ii \omega t} \delta(\vec x)$, one frequency of $c^2 \delta(\vec x) \delta(t)$, the solution using only outgoing waves would be
\be
\hat g(r, \omega) = \frac{\ee^{\ii k r}}{4 \pi r},
\en
 with $k = \omega /c$, which after a Fourier transform would give the causal 3D Greens function~\eqref{eqn:ScalarGreen3D} as expected.

For the frequency decomposition of the 2D Greens function $\hat g_2$, we imagine that all functions will be independent of the $z$ coordinate. So to use $\hat g$ in a convolution we need to first integrate over $z$:
\be
\hat g_2 = \int_{-\infty}^\infty \hat g(r, \omega) dz =  \int_{-\infty}^\infty \frac{\ee^{\ii k \sqrt{ r^2 + z^2} }}{4 \pi \sqrt{ r^2 + z^2}} dz = \frac{\ii}{4}H_0^{(1)}(k r),
\en
where $r^2 = x^2 + y^2$ and $\hat g_2$ is an outgoing wave solution to
\be
 k^2 \hat g_2 + \nabla^2_2 \hat g_2 = - \delta(x)\delta(y),
\en
where $\hat B(\vec x , \omega) = c^2 \delta(\vec x)$ and $\nabla_2$ is the gradient in $x$ and $y$. Note that Graf's and Gegenbauer's addition formulas are very useful to rewrite any bessel or hankel function.

To calculate $g_2$ we can take the 3D Greens~\eqref{eqn:ScalarGreen3D} substitute $r \to \sqrt{r^2 + z^2}$ and integrate in $z$ to get
\be
g_2 = \int_{-\infty}^{\infty} \frac{\delta(c t - \sqrt{r^2 + z^2})}{4 \pi \sqrt{r^2 + z^2}} c dz = \frac{c}{2 \pi} \frac{H_s(t -|r|/c)}{\sqrt{c^2 t^2 - r^2}},
\en
where $H_s$ is the Heavside step function, so that $H_s(t-|r|/c)$ is zero for $r > ct$.

We can now calculate the 2D incident wave for $B(\vec x,t) = \delta(\vec x) b (t)$ by using the procedure~\eqref{eqn:ScalarPointImpact} for $g_2$ to find
\be
\varphi^I(\vec x,t) =
 \int_{-\infty}^{\infty} g_2(r ,t -\tau) \frac{b(\tau)}{c^2}  d\tau = \int_{|r|/c}^{\text{max} \{t, \frac{|r|}{c}\} } \frac{1}{2 \pi c} \frac{b(t-\tau)}{\sqrt{c^2 \tau^2 - r^2}}  d\tau,
  %\int_{-\infty}^{t-r/c}  \frac{1}{2 \pi c} \frac{b(\tau) }{\sqrt{c^2 (t- \tau)^2 - r^2}}  d\tau,
\label{eqn:ScalarPointImpact2D}
\en
where we changed variables $\tau \leftarrow t -\tau$ so that we can differentiate the above expression in $t$ more easily (specially numerically) and assumed that $b(-t) =0$ for $t>0$.
Alternatively,  Eq.~\eqref{eqn:ScalarPointImpact2D} can also be written in terms of the Fourier transforms $\hat g_2(r, \omega)$ and $\hat b(\omega)$ as
\be
\varphi^I(\vec x,t) =
 \frac{1}{c^2} \int_{-\infty}^{\infty}  g_2(r ,\tau) b(t -\tau)  d\tau = \frac{1}{2 \pi c^2} \int_{-\infty}^{\infty} \ee^{-\ii \omega t}  \hat g_2(r ,\omega) \hat b(\omega)  d \omega.
\label{eqn:ConvolutionWithGreen2D}
\en

Assuming that $b(t) = 0$ for $t \not \in [0, T]$, then we turn into a numerical method by approximating $b(t)$ with its truncated Fourier series $b_N(t)$ so that
\be
b_N(t) = \frac{1}{T} \sum_{n = -N}^{ N} \hat b_N^n \ee^{- \ii t \frac{2 \pi n}{T}}, \;\; \text{where} \;\; \hat b_N^n := \hat b_N\left (\frac{2 \pi n}{T}\right ) =  \int_0^T b(t) \ee^{\ii \frac{2 \pi n t}{T}} dt,
\en
%which can be used to approximate the Fourier transformed $\hat b(\omega)$ for $\omega \in [- 2 \pi \frac{N}{T}, 2 \pi \frac{N}{T}]$ using
%\be
%c_n = \int_{0}^T b_N(t) \ee^{\ii t \frac{2 \pi n}{T} } dt  = \hat b_N(2 \pi n/T) ,
%\en
%for $n= - N, -N+1, \ldots, N$.
 We can turn this into the discrete Fourier transform by using only the points
 %$b_N(m T /(2N +1))$ with $m=0, 1, \ldots 2N$, from which we get
\be
b_N^m := b_N \left (\frac{m T}{2N +1}\right )= \frac{1}{T} \sum_{n = -N}^{ N} \hat b^n_N \ee^{-2 \pi  \ii  \frac{ m n}{2N +1}} =  \frac{1}{T} \sum_{n = 0}^{ 2 N}  \hat b^{n- N}_N  \ee^{-2 \pi  \ii  \frac{ m n}{2 N}}   \ee^{2 \pi  \ii m  \frac{N}{2 N +1}},
\en
for $m= 0, 1, \ldots, 2 N$. We can now apply some linear algebra to extract the coefficients $T^{-1}\hat b^{n- N}_N$ of the vectors
\be
(\vec v_n)^m:=  \ee^{-2 \pi  \ii  \frac{ m n }{2 N + 1}} \ee^{2 \pi  \ii m  \frac{N}{2 N +1}} \text{, where } \; \vec v_n \cdot \bar{\vec v}_j = (2 N +1) \delta_{nj},
\en
to reach that
\be
\hat b^{n}_N  = \frac{T}{(2 N +1)} \sum_{m=0}^{2 N} b_N^m \ee^{2 \pi  \ii  \frac{ m n }{2 N + 1}},
\en
which is the definition of the discrete Fourier transform.

The convolution formula~\eqref{eqn:ConvolutionWithGreen2D} can now be approximated by
%\be
%\varphi^I(\vec x,t) \approx
 %\frac{1}{c^2} \int^{t}_{\text{min}\{|r|/c,\, t-T\}} g_2(r ,\tau) b_N(t -\tau)  d\tau \approx \frac{1}{2 \pi c^2} \int_{- 2 \pi N/T}^{ 2 \pi N/T} \ee^{-\ii \omega t}  \hat g_2(r ,\omega) \hat b_N(\omega)  d \omega.
%\label{eqn:ConvolutionWithGreen2DN}
%\en
\be
\varphi^I(\vec x,t) \approx \frac{1}{T c^2}   \sum_{n = - N}^N \ee^{-\ii \frac { 2 \pi n }{T} t}  \hat g_2(r ,2 \pi n /T) \hat b_N^n   .
\label{eqn:ConvolutionWithGreen2DN}
\en
To choose $N=20$ for the package add \texttt{"FrequencyModes"->20} to the function call. One possible issue is that $\hat g_2$ has a singularity at $\omega =0$. More generally, every Hankel function of the first type has a singularity at $\omega =0$,  which we will deal with carefully in the next section.
%so we must make sure that $\hat b_N(\omega)$ is bounded and preferably smooth near $\omega =0$.

\subsection{Scattered wave}

The Fourier transform of the outgoing waves from a cylinder can be anything in the form
\be
\hat \varphi^S(r, \omega) = \sum_{n= -\infty}^{ \infty} a_n H^{(1)}_n( \omega r/c) \ee^{\ii n \theta} \approx  \sum_{n= -Na}^{ Na} a_n H^{(1)}_n( \omega r/c) \ee^{\ii n \theta}
\label{eqn:ScatteredForm}
\en
where we call $Na$ the number of angular modes. To choose $Na=6$ for the package add \texttt{"NAngularModes"->6} to the function call.

To calculate the scattered wave in time $\varphi^S$, we do the same operation on $\hat \varphi_S$ that we did on $\hat g_2$ in Eq.\eqref{eqn:ConvolutionWithGreen2DN}
\begin{multline}
 \varphi^S(r, t)  \approx \frac{1}{2 \pi c^2} \int_{- 2 \pi N/T}^{ 2 \pi N/T} \ee^{-\ii \omega t}  \hat \varphi^S(r ,\omega) \hat b_N(\omega)  d \omega
 \\
    = \frac{1}{2\pi c} \sum_{n= -\infty}^{ \infty}  \int_{- 2 \pi N/T}^{ 2 \pi N/T}   \ee^{ -\ii c k t + \ii n \theta } a_n(k) H^{(1)}_n( k r)  \hat b_N(c k)  d k.
\label{eqn:ScatteredConvolutionN}
\end{multline}
The coefficients $a_n$ will be determined by the boundary conditions.


\subsubsection{Boundary condition}

To compare the incident and scattered wave we use Graf's addition formula~\footnote{see \url{http://dlmf.nist.gov/10.23\#E7}} applied to
\be
H^{(1),(2)}_0 (k \|\vec r- \vec r_I \|) = \sum_{n=-\infty}^{\infty} \ee^{\ii n (\theta -\theta_1)} J_n (k r) H^{(1),(2)}_n(k r_I)  \quad \text{if} \;\; r< r_I,
\en
%~\eqref{eqn:ScatteredWave} we rewrite\footnote{See obscure reference on p.159 of Pao and Mow, and use $J_{n}= (-1)^n J_{-n}$ together with $H_{n}= (-1)^n H_{-n}$.}
%\be
%\hat g_2 (r,r_I) =  (1 + \delta_m^0)\frac{\ii}{8} \sum_{m=-\infty}^\infty \ee^{\ii m(\theta-\theta_1)}
%\begin{cases}
    %J_m(k r_I) H_m^{(1)}(k r),&  r > r_I,\\
%J_m(k r) H_m^{(1)}(k r_I),&  r < r_I,
%\end{cases}
%\en
where the body force originates at $\vec r_I = r_I (\cos \theta_I,\sin \theta_I)$  and $\vec r = r (\cos \theta,\sin \theta)$.
 %and $\delta_m^0 =1$ only if $m=0$.
 If $\hat \varphi^I = \hat g_2( \|\vec r- \vec r_I\|)$ and $\hat \varphi^S =  a_{ n} H^{(1)}_n (k r) \ee^{\ii n \theta}$ then the boundary condition  $\hat \varphi_I + \hat \varphi_S =0$ on $r= r_S< r_I$ implies
%\varphi^S= $\sum_{n=-\infty}^\infty a_{ n} H^{(1)}_n (k r) \ee^{-\ii k c t} \ee^{\ii n \theta}$
\bga
\frac{\ii}{4} \ee^{- \ii n \theta_I} J_n (k r_S) H^{(1)}_n(k r_I) + a_n H_n^{(1)}(k r_S) =0 \implies
\\
 a_n  =-\frac{\ii}{4} \ee^{- \ii n \theta_I} \frac{J_n (k r_S)}{H_n^{(1)}(k r_S)} H^{(1)}_n(k r_I).
 \label{eqn:BCDirichlet}
\ega
while the boundary condition $\partial \hat \varphi^I /\partial r + \partial \hat \varphi^S/ \partial r =0$ on $r= r_S < r_I$ implies that
\bga
% a_n k H^{'(1)}_n (k r_S) + (1+\delta_n^0) \frac{i \ee^{\ii n \theta_I}}{8}  k J_n'(k r_S) H_n^{(1)}(k r_I) =0
%\implies
%\notag \\
a_n  =- \frac{i}{4} \ee^{- \ii n \theta_I}  \frac{ J_n'(k r_S)}{ H^{'(1)}_n (k r_S)} H_n^{(1)}(k r_I).
\label{eqn:BCNeumann}
\ega

\subsection{Wave frequency to time response}

This module is used for more complicated scattering, such as MST. Any outgoing wave from a scatterer can be expanded as
\[
\hat \psi^{S}(\vec r, k) \approx \sum_{n=-Na}^{Na} a_{n} H_n (k r) \ee^{\ii n \theta},
\]
for a cylindrical coordinate system with origin at the scatterer, where $H_n:= H_n^{(1)}$ a Hankel function of the first kind. To recover the wave in time we need to approximate the inverse Fourier transform:
\[
\psi^{S}(\vec r ,t ) = \frac{c}{2 \pi}\int_{-\infty}^\infty \hat \psi^{S}(\vec r ,k ) \ee^{-\ii  c k t} d k  \approx \frac{c}{2 \pi} \sum_{n=-Na}^{Na}\ee^{\ii n \theta}  \int_{- 2 \pi N /T}^{2 \pi N /T}  a_{n} H_n (k r) \ee^{-\ii  c k t} d k,
\]
where $T$ is the period of the incident wave and $N$ is the number of frequency modes.

Assuming that the scatterers are small compared with the wavelength $k r_S << 1$, where $r_S$ is the radius of the scatterer, then the most general form for outgoing waves from the $j$-th scatterer is
\bga
\psi^{S}(\vec r) = a H_0 (k r) + (c \cos \theta + s \sin \theta ) H_1 (k r) ,
\ega
which means that $a_0 = a$, $a_1 = (s + c)/4$ and  $a_{-1} = (s-c)/4$.


This package was used in the talk:


Gower, Artur L., et al. \emph{"Characterizing composites with acoustic backscattering: Combining data driven and analytical methods."} The Journal of the Acoustical Society of America 141.5 (2017): 3810-3810.




\end{document}
